\documentclass{unswmaths}

\usepackage{unswshortcuts}

\begin{document}

\subject{Measure Theory}
\author{Edward McDonald}
\title{Homework 4}
\studentno{3375335}


\setlength\parindent{0pt}


\newcommand{\Bor}{\mathcal{B}(\Rl)}
\newcommand{\sdiff}{\bigtriangleup}

\unswtitle{}


\section*{Question 1}
For this question, $(X,\mathcal{A},\mu)$ is a measure space.
\begin{lemma}
    The set 
    \begin{equation*}
        \{[a,\infty)\;:\;a \in\Rl\}
    \end{equation*}
    generates $\mathcal{B}(\Rl)$.
\end{lemma}
\begin{proof}
    Since $[a,b) = [c,\infty)^c\cap [a,\infty)$, by the result
    of homework week 2, this set generates $\mathcal{B}(\Rl)$.
\end{proof}
Hence to prove that $f:X\rightarrow\Rl$ is Borel
measurable, it is sufficient to show that $f^{-1}([a,\infty)) \in \mathcal{A}$
for all $a \in \Rl$.

\begin{lemma}
\label{fnsonR}
     If $f,g:X\rightarrow \Rl$ are Borel measurable functions, then $f+g$ is Borel
     measurable.
\end{lemma}
\begin{proof}
     Let $a \in \Rl$. Then for every $r \in \mathbb{Q}$, we
     have $f(x) + g(x) \geq a$ if and only if $f(x) \geq r$ and $g(x) \geq a-r$. Hence,
     \begin{equation*}
         (f+g)^{-1}([a,\infty)) = \bigcup_{r \in \mathbb{Q}} \{ f^{-1}([r,\infty])\cap g^{-1}([a-r,\infty))\}
     \end{equation*}
     Since $f$ and $g$ are Borel measurable, the right hand side is in $\mathcal{A}$.
     Hence $f+g$ is Borel measurable.
\end{proof}
\begin{lemma}
    The following functions on $\Rl$ are Borel measurable:
    \begin{itemize}
        \item{} $s_1(x) = x^2$
        \item{} $s_2(x) = \alpha x$, for any $\alpha \in \Rl$
        \item{} $s_3(x) = x^{-1}$, is measurable on the subspace $\Rl\setminus\{0\}$.
        \item{} $s_4(x) = |x|$
    \end{itemize}
\end{lemma}
\begin{proof}
    Let $a \in \Rl$. If $a < 0$, then $s_1^{-1}([a,\infty)) = \Rl$ and otherwise $s_1^{-1}([a,\infty)) = (-\infty,-\sqrt{a})\cup(\sqrt{a},\infty)$.
    Hence $s_1$ is Borel.
    
    For $s_2$, the case $\alpha = 0$ is trivial. Suppose $\alpha > 0$. Then $s_2^{-1}([a,\infty)) = [a/\alpha,\infty)$. Similarly,
    if $\alpha < 0$, $s_2^{-1}([a,\infty)) = (-\infty, a/\alpha)$. Hence $s_2$ is Borel for any $\alpha$.
    
    Now if $a > 0$, $s_3^{-1}([a,\infty)) = (0,1/a)$ and if $a = 0$ $s_3^{-1}([a,\infty)) = (0,\infty)$.
    
    If $a < 0$, $s_3^{-1}([a,\infty)) = (-\infty,1/a)\cup (0,\infty)$.
    
    Hence $s_3$ is Borel since the set $(0,1/a)$, $(0,\infty)$, $-\infty,1/a)$ and $(0,\infty)$ are
    Borel on the subspace $\Rl\setminus \{0\}$.
    
    Now $s_4^{-1}([a,\infty)) = (-\infty,-a]\cup[a,\infty)$ for any $a \in \Rl$, and 
    so $s_4$ is measurable.
\end{proof}

\begin{corollary}
    Let $f:X\rightarrow\Rl$ be measurable. The following functions on $X$ are measurable:
    \begin{itemize}
        \item{} $f_1 = f^2$
        \item{} $f_2 = \alpha f$ for any $\alpha \in \Rl$.
        \item{} $f_3 = 1/f$, measurable on the subspace of $X$ given by $X\setminus\{x\;:\;f(x) = 0\}$.
        \item{} $f_4 = |f|$.
    \end{itemize}
\end{corollary}
\begin{proof}
    We see that $f_1$, $f_2$ and $f_4$ are measurable since $f_1 = s_1\circ f$, $f_2 = s_2 \circ f$
    and $f_4 = s_4\circ f$ using
    $s_1$, $s_2$ and $s_4$ from lemma \ref{fnsonR}.
    
    For $f_3$, note that $X\setminus\{x\;:\; f(x) = 0\} = f^{-1}(\Rl\setminus \{0\})$, and by definition
    subsets of $X\setminus \{x\;:\;f(x)=0\}$ are measurable if they are the intersection with some element
    of $\mathcal{A}$.
    
    See that since $f_3 = s_3 \circ f$, it is measurable.
\end{proof}

\begin{lemma}
    If $f,g:X\rightarrow\Rl$ are measurable, then so is $fg$.
\end{lemma}
\begin{proof}
    We may write $fg = \frac{1}{2}((f+g)^2-f^2-g^2)$. Hence $fg$ may be written as a composition of measurable functions
    and is hence measurable.
\end{proof}

\begin{corollary}
    If $f$ and $g$ are measurable functions on $X$, then $f/g$ is measurable on $X\setminus\{x\;:\;g(x) = 0\}$.
\end{corollary}
\begin{proof}
    Since $f$ is measurable on $X$, it is measurable on $X\setminus\{x\;:\;g(x)=0\}$. Hence since $1/g$ is measurable
    on this set, their product is measurable.
\end{proof}

\begin{lemma}
    If $f,g:X\rightarrow\Rl$ are measurable, then so is $\max\{f,g\}$ and $\min\{f,g\}$.
\end{lemma}
\begin{proof}
    We may write $\max\{f,g\} = \frac{1}{2}(f+g+|f-g|)$ and $\min\{f,g\} = \frac{1}{2}(f+g-|f-g|)$. Hence
    these are compositions of measurable functions and so measurable.
\end{proof}

\begin{lemma}
    If $f,g:X\rightarrow\Rl$ are measurable, then the sets $\{x\;:\;f(x) = g(x)\}$,
    $\{x\;:\;f(x) > g(x)\}$ and $\{x\;:\;f(x) \geq g(x)\}$ are measurable.
\end{lemma}
\begin{proof}
    We may write these sets respectively as $(f-g)^{-1}(\{0\})$,
    $(f-g)^{-1}((0,\infty))$ and $(f-g)^{-1}([0,\infty))$.
    Hence since $f-g$ is measurable, the result follows.
\end{proof}

\begin{lemma}
    If $\{f_n\}_{n=1}^\infty$ is a sequence of measurable functions on $X$, then so is
    $\inf_{n} f_n$.
\end{lemma}
\begin{proof}
    Let $f = \inf_{n} f_n$ and $a \in \Rl$. Let $x \in f^{-1}([a,\infty))$.
    Then for all $n$, $f_n(x) \geq a$, so $x \in f_n^{-1}([a,\infty))$.
    Hence,
    \begin{equation*}
        f^{-1}([a,\infty)) \subseteq \bigcap_{n=1}^\infty f_n^{-1}([a,\infty))
    \end{equation*}
    Similarly, if $f_n(x) \geq a$ for all $n$, then we must have $f(x) \geq a$. Hence,
    \begin{equation*}
        f^{-1}((-\infty,a]) = \bigcap_{n=1}^\infty f_n^{-1}([a,\infty)).
    \end{equation*}
    Since each $f_n$ is measurable, the right hand side is measurable. Hence $f$ is measurable.
\end{proof}
\begin{corollary}
    If $\{f_n\}_{n=1}^\infty$ is a sequence of measurable real valued functions on $X$, then
    the following functions are measurable:
    \begin{itemize}
        \item{} $\sup_{n} f_n$
        \item{} $\limsup_{n} f_n$
        \item{} $\liminf_{n} f_n$
    \end{itemize}
\end{corollary}
\begin{proof}
    We may write $\sup_{n} f_n = -\inf_{n} -f_n$, hence it is measurable.
    
    $\limsup_n = \inf_{n} \sup_{k\geq n} f_k$ and $\liminf_{n} = \sup_{n}\inf_{k\geq n} f_n$.
    Hence these functions are measurable.
\end{proof}

\section*{Question 2}
    For this question, we consider the measure space $(X,\mathcal{A},\mu)$
    and a bounded measurable measurable function $f:X\rightarrow [0,\infty)$.
    
    Define $F:[0,\infty)\rightarrow[0,\infty]$
    by $F(t) = \mu(\{x\;:\;f(x) > t\})$.
    
\begin{lemma}
    $F$ is a decreasing function and $F$ vanishes outside some bounded interval.
\end{lemma}
\begin{proof}
    It is clear that $F$ is a decreasing function, since if $s > t$, 
    \begin{equation*}
        \{x\;:\;f(x) > s\} \subseteq \{x\;:\;f(x) > t\}.
    \end{equation*}
    Hence $\mu\{x\;:\;f(x) > s\} \leq \mu\{x\;:\;f(x) > t\}$, so $F(s) \leq F(t)$.
    
    Since $f$ is bounded, there is some $M$ such that $f(x) < M$ for all $x$. Hence for $x > M$, $F(x) = \mu(\emptyset) = 0$.
    Thus the support of $F$ is contained in $[0,M]$.
\end{proof}

\begin{lemma}
    The limit $\lim_{t\rightarrow 0} F(t)$ is finite if and only if the support of $f$ has finite measure.
\end{lemma}
\begin{proof}
    We may write $\lim_{t\rightarrow 0} F(t) = \lim_{n\rightarrow \infty} \mu\{x\;:\;F(x) > 1/n\} = \mu(\bigcap_{n=1}^\infty \{x\;:\;F(x) > 1/n\}) = \mu(\{x\;:\;F(x) > 0\})$.
    
    Hence this limit is finite if and only if $F$ is nonzero on a set of finite measure.
\end{proof}
\begin{theorem}
    The indefinite Riemann integral,
    \begin{equation*}
        \int_0^\infty F(t) \;dt
    \end{equation*}
    exists if $f$ has support of finite measure.
\end{theorem}
\begin{proof}
    Since $F$ vanishes outside some interval $[0,M]$, we have that
    \begin{equation*}
        \lim_{N\rightarrow \infty} \int_{0}^N F(t)\;dt = \int_{0}^M F(t)\;dt.
    \end{equation*}
    Hence we need only consider the integrability of $F$ on $[0,M]$.
    
    Since $f$ has support of finite measure, and $F$ is a decreasing function, we have that $F$ is bounded.
    
    Hence $F$ is a bounded monotone function, so the Riemann integral exists.
\end{proof}

Suppose $f$ has support of finite measure and let $F$ have support contained in $[0,N]$. 
Consider the partition of $[0,N]$ given by $\mathcal{P}_n = \{0,1/2^n,2/2^n,\ldots,N2^n/2^n\}$.
Let $L_n$ be the corresponding lower Riemann sum of $F$.

\begin{lemma}
    $L_n$ is given by the integral of a simple function bounded above by $F$, $s_n$
    given by
    \begin{equation*}
        s_n = \sum_{k=0}^{N-1} \chi_{f^{-1}([k/2^n,(k+1)/2^n))} F(k/2^n).
    \end{equation*}
\end{lemma}
\begin{proof}
    Since $F$ is a decreasing function, the infimum of $F$ on the interval $[k/2^n,(k+1)/2^n)$
    is $F((k+1)/2^n)$.
    Hence,
    \begin{equation*}
        L_n = \sum_{k=0}^{N-1} F((k+1)/2^n)\frac{1}{2^n}
    \end{equation*}
    
    
    
\end{proof}
\begin{proof}
    
\end{proof}


\end{document}
