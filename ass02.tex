\documentclass{unswmaths}

\usepackage{unswshortcuts}

\begin{document}

\subject{Measure Theory}
\author{Edward McDonald}
\title{Homework 2}
\studentno{3375335}


\setlength\parindent{0pt}


\newcommand{\Bor}{\mathcal{B}(\Rl)}


\unswtitle{}


\section*{Question 1}
\subsection*{(a)}
    We are given,
    \begin{equation*}
        \mathcal{C} = \{\emptyset, X, \{1\},\{2,3\},\{1,2,3\},\{4,5\}\}.
    \end{equation*}
    See that $\mathcal{C}$ is closed under intersection, so we only
    need to consider unions of elements of $\mathcal{C}$
    Hence, 
    \begin{equation*}
        \sigma(\mathcal{C}) = \{\emptyset, X, \{1\},\{2,3\},\{4,5\},\{1,2,3\},\{1,4,5\},\{2,3,4,5\}\}.
    \end{equation*}
    
    
\subsection*{(b)}
Then we define $\gamma:\mathcal{C}\rightarrow[0,\infty]$ as follows,
\begin{align*}
    \gamma(\emptyset) = 0\\
    \gamma(\{1\}) = 1\\
    \gamma(\{2,3\}) = 1\\
    \gamma(\{1,2,3\}) = 2\\
    \gamma(\{4,5\}) = 1\\
    \gamma(X) = 3.
\end{align*}
Let $\mu^*$ be the outer measure on $2^X$ defined by $\gamma$.
\begin{lemma}
    $\mu^*$ restricted to $\sigma(\mathcal{C})$ is a measure.
\end{lemma}
\begin{proof}
    Since $\mu^*$ is a measure on $\mathcal{M}(\mu^*)$, it is sufficient to show
    that $\mathcal{C} \subseteq \mathcal{M}(\mu^*)$. Then since $\mathcal{M}(\mu^*)$ is
    a $\sigma$-algebra, $\sigma(\mathcal{C}) \subseteq \mathcal{M}(\mu^*)$.
    
    Clearly $\emptyset$ and $X$ are in $\mathcal{M}(\mu^*)$, so 
    we need only show that $\{1\},\{2,3\},\{1,2,3\}$ and $\{4,5\}$
    are in $\mathcal{M}(\mu^*)$. We do not need to show that $\{1,2,3\} \in \mathcal{M}(\mu^*)$
    since $\{1,2,3\} = \{1\}\cup\{2,3\}$.
    
    Let $A \subset X$. Then
    \begin{equation*}
        \mu^*(A) = \mu^*(A\cap \{1\}) + \mu^*(A\setminus\{1\})
    \end{equation*}
    since if $1 \in A$, $\mu^*(A\cap\{1\}) = 1$ and $\mu^*(A\setminus\{1\})$,
    
\end{proof} 

\section*{Question 2}
For this question, $\mu^*$ is an outer measure on the set $X$ defined by the function $\gamma:\mathcal{C}\subseteq 2^X\rightarrow[0,\infty]$.
\begin{lemma}
    $\mu^*(\emptyset) = 0$.
\end{lemma}
\begin{proof}
    By definition,
    \begin{equation*}
        \mu*(\emptyset) = \inf\{\sum_{n=0}^\infty \gamma(A_n) \;:\; \emptyset \subseteq \bigcup_{n=0}^\infty A_n,A_n \in \mathcal{C}\;\}.
    \end{equation*}
    However this is simply
    \begin{equation*}
        \mu^*(\emptyset) = \inf\{\gamma(A)\;A \in \mathcal{C}\}
    \end{equation*}
    since $\emptyset$ is a subset of any set. However $\gamma(\emptyset) = 0$,
    so $\mu^*(\emptyset) = 0$.
\end{proof}

\begin{lemma}
    If $A \subseteq B \subseteq X$, then $\mu^*(A) \leq \mu^*(B)$.
\end{lemma}
\begin{proof}
    Since $A \subseteq B$, we have the set inclusion
    \begin{equation*}
        \{\{E_i\}_{i=0}^\infty \subseteq \mathcal{C}\;:\;A \subseteq \bigcup_{i=0}^\infty E_i\} \subseteq \{\{E_i\}_{i=0}^\infty \subseteq \mathcal{C}\;:\;B \subseteq \bigcup_{i=0}^\infty E_i\}
    \end{equation*}
    Hence,
    \begin{equation*}
        \inf\{\sum_{i=0}^\infty \gamma(E_i) \subseteq \mathcal{C}\;:\;A \subseteq \bigcup_{i=0}^\infty E_i\} \leq \inf\{\sum_{i=0}^\infty \gamma(E_i) \subseteq \mathcal{C}\;:\;B \subseteq \bigcup_{i=0}^\infty E_i\}
    \end{equation*}
    So by definition,
    \begin{equation*}
        \mu^*(A)\leq\mu^*(B).
    \end{equation*}
\end{proof}

\section*{Question 3}
The euclidean topology $\tau$ on $\Rl$ is defined as the toplogy generated
by open intervals $(a,b)$,$a < b \in \Rl$
\begin{lemma}
    Every open set in $\tau$ is a countable union of intervals $(a,b)$,$a,b\in \Rl$.
\end{lemma}
\begin{proof}
    Let $U \in \tau$, and $x \in U$. Then there is some $\varepsilon > 0$
    such that $(x-\varepsilon,x+\varepsilon) \subset U$. By the density
    of $\mathbb{Q}$ in $\mathbb{R}$, there is some $p,q \in \mathbb{Q}$
    such that $q \in (x-\varepsilon,x)$ and $q \in (x,x+\varepsilon)$.
    Then $x \in (p,q) \subset U$. 
    
    Hence $U$ can be expressed as a union of intervals of the form $(p,q)$
    for $p,q \in \mathbb{Q}$. 
    
    However the set $\{(p,q)\;:\;p,q \in \mathbb{Q}\}$ is countable, so
    any open set in $\tau$ is a countable union of intervals.
\end{proof} 
\begin{theorem}
    The set $\mathcal{T}$ consisting of all countable unions of open 
    intervals $(a,b)$, $a,b \in \mathbb{R}$ is a topology.
\end{theorem}   
\begin{proof}
    Since any open set in $\tau$ is a countable union of intervals, $\tau \subseteq \mathcal{T}$,
    and since any countable union of intervals is in $\tau$, $\mathcal{T} \subseteq \tau$.
    Hence $\mathcal{T} = \tau$.
\end{proof}

We now consider the Borel $\sigma$-algebra $\Bor = \sigma(\mathcal{T})$.
\begin{theorem}
    $\Bor$ can be generated by the systems
    \begin{align*}
        \mathcal{E}_1 = \{(a,b)\;:\;a,b \in \Rl\},\\
        \mathcal{E}_2 = \{[a,b]\;:\;a,b \in \Rl\},\\
        \mathcal{E}_3 = \{(a,b]\;:\;a,b \in \Rl\},\\
        \mathcal{E}_4 = \{[a,b)\;:\;a,b \in \Rl\},\\
        \mathcal{E}_5 = \{(a,b)\;:\;a,b \in \mathbb{Q}
    \end{align*}
\end{theorem}
\begin{proof}
    Since $\mathcal{E}_1 \subseteq \mathcal{T}$, we have $\sigma(\mathcal{E}_1) \subseteq \sigma(\mathcal{T}) = \Bor$.
    
    However since any element of $\mathcal{T}$ is a countable union of elements of $\mathcal{E}_1$,
    we must have $\mathcal{T} \subseteq \sigma(\mathcal{E}_1)$, hence $\Bor \subseteq \sigma(\mathcal{E}_1)$.
    
    Now let $a,b \in \Rl$, and consider the sequence of sets $\{[a+1/n,b-1/n]\}_{n=1}^\infty \subset \mathcal{E}_2$. Hence $\bigcup_{n=1}^\infty [a-1/n,b+1/n] = (a,b) \in \sigma(\mathcal{E_2})$.
    
    Hence $\mathcal{E}_1 \subset \sigma{\mathcal{E}_2)$, so $\sigma(\mathcal{E}_1) \subseteq \sigma(\mathcal{E}_2)$.
    
    Similarly, $[a,b] = \bigcap_{n=1}^\infty (a-1/n,b+1/n) \in \sigma(\mathcal{E}_1)$. 
    Thus $\mathcal{E}_2 \subseteq \sigma(\mathcal{E}_1)$. Hence $\sigma(\mathcal{E}_1) = \sigma(\mathcal{E}_2}$.
    
    Similarly, by considering the sequences of sets $(a,b+1/n]$, $(a,b-1/n]$, $[a-1/n,b)$
    and $[a+1/n,b)$ we can show that $\sigma(\mathcal{E}_3) = \sigma(\mathcal{E}_4} = \sigma(\mathcal{E}_1) = \Bor$.
    
    
    Finally, if $a < b \in \Rl$, and $x \in (a,b)$ there exist rational
    numbers $p$ and $q$ such that $x \in (p,q) \subset (a,b)$. Hence 
    $(a,b)$ is a union of intervals with rational endpoints,
    and so a countable union of elements of $\mathcal{E}_5$.
    This $\mathcal{E}_1 \subseteq \sigma(\mathcal{E}_5)$. Hence
    $\sigma(\mathcal{E}_1) \subseteq \sigma(\mathcal{E}_5)$.
    Since $\mathcal{E}_5 \subseteq \mathcal{E}_1$, we conclude
    $\sigma(\mathcal{E}_5) = \Bor$.
    
\end{proof}

\section*{Question 4}
We define $\mathcal{M}$ as the smallest family of sets containing $\mathcal{T}$
which is closed under countable intersections and unions.

This is well defined since $2^\Rl$ is a family of sets containing $\mathcal{T}$
which is closed under countable intersections and unions. Hence the expression
\begin{equation*}
    \mathcal{M} = \bigcap \mathcal{A}\subset 2^\Rl \mathcal{A}
\end{equation*}
as $\mathcal{A}$ varies over all subsets of $2^\Rl$ containing $\mathcal{T}$
closed under countable unions and intersections is well defined since the right hand
side is the intersection of nonempty family of sets.

\begin{lemma}
    Any closed set in $\Rl$ is a countable intersection of open sets.
\end{lemma} 
\begin{proof}
    Let $F \subset \Rl$ be any subset. Define
    \begin{equation*}
        F_\varepsilon = \bigcup_{x \in F} (x-\varepsilon, x+\varepsilon) \in \mathcal{T}.
    \end{equation*}
    Let $F_0 = \bigcap_{n=1}^\infty F_{1/n}.
    Suppose $F \subseteq V$, where $V$ is closed.
\end{proof}




\end{document}
