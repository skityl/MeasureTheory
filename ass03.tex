\documentclass{unswmaths}

\usepackage{unswshortcuts}

\begin{document}

\subject{Measure Theory}
\author{Edward McDonald}
\title{Homework 3}
\studentno{3375335}


\setlength\parindent{0pt}


\newcommand{\Bor}{\mathcal{B}(\Rl)}
\newcommand{\sdiff}{\bigtriangleup}

\unswtitle{}

\section*{Question 1}
In this question we work over $\Rl^d$, with $d > 1$. $\lambda_d$ 
is the Lebesgue measure on $\Rl^d$.
\begin{theorem}
    Let $\ell \subset \Rl^d$ be a line. Then $\lambda_d(\ell) = 0$.
\end{theorem}   
\begin{proof}
    Since $\ell$ is closed, $\ell$ is Borel and hence Lebesgue measurable.
    
    Since $\lambda_d$ is translation invariant, we may assume that 
    $0 \in \ell$. Let $v \in \ell$ with $v \neq 0$. Consider half open line segment,
    \begin{equation*}
        S = \{tv\;:\;t \in [0,1)\;\}.
    \end{equation*}
    $S$ is Borel, hence measurable. Suppose $v = (v_1,v_2,\ldots,v_d)$, and define the box,
    \begin{equation*}
        B_n = \prod_{k=1}^d [0,\frac{v_k}{2^{n}}).
    \end{equation*}
    If $t \in [0,1)$ then for any $n > 0$ there exists
    an integer $k$ such that $\frac{k}{2^n} \leq v \leq \frac{k+1}{2^n}$.
    
    Hence,
    \begin{equation*}
        S \subset \bigcup_{k=0}^{2^n-1} (B_n+\frac{k}{2^{n}}v).
    \end{equation*}
    So by translation invariance,
    \begin{equation*}
        \lambda_d(S) \leq 2^n \lambda_d(B_n).
    \end{equation*}
    But by definition, $\lambda_d(B_n) = \frac{1}{2^{nd}}\lambda_d(B_0)$.
    Hence, 
    \begin{equation*}
        \lambda_d(S) \leq 2^{n(1-d)}\lambda_d(B_0).
    \end{equation*}
    Since $d > 1$, $n$ is arbitrary and $\lambda_d(B_0)$ is finite, we
    conclude that $\lambda_d(S) = 0$.
    
    Now since
    \begin{equation*}
        \ell = \bigcup_{n\in \mathbb{Z}} (nv+S)
    \end{equation*}
    by translation invariance we conclude that $\lambda_d(\ell) = 0$.    
\end{proof}

\section*{Question 2}
Let $A$,$B$ and $C$ be sets, and define $A \sdiff B := (A\setminus B)\cup (B\setminus A)$.
\subsection*{(a)}
\begin{lemma}
    $A^c \sdiff B^c = A\sdiff B$.
\end{lemma}
\begin{proof}
    This is a simple computation, since by definition,
    \begin{align*}
        A^c \sdiff B^c &= (A^c \setminus B^c)\cup (B^c \setminus A^c)\\
        &= (A^c\cap B)\cup (B^c \cap A)\\
        &= (B\cap A^c)\cup (A\cap B^c)\\
        &= (B \setminus A)\cup (A\setminus B)\\
        &= (A \sdiff B).
    \end{align*}
\end{proof}
\begin{lemma}
    $A\sdiff C \subseteq (A\sdiff B)\cup (B\sdiff C)$
\end{lemma}
\begin{proof}
    See that
    \begin{align*}
        A\sdiff C &= (A\sdiff C)\cap B \cup (A\sdiff C)\cap B^c\\
        &= (A\cap B\cap C^c)\cup (A^c \cap B \cap C)\cup (A\cap B^c \cap C^c)\cup (A^c\cap B^c\cap C)\\
        &\subseteq (A^c\cap B)\cup (B^c\cap A)\cup (B\cap C^c)\cup (B^c\cap C)\\
        &= (A\sdiff B)\cup (B\sdiff C).
    \end{align*}
\end{proof}

\subsection*{(b)}
Now we define
\begin{equation*}
    \mathcal{G} := \{\;B \in \mathcal{F}\;:\;\forall\varepsilon > 0\;\exists B_\varepsilon \in \mathcal{A}\;\text{such}\;\text{that}\;\mu(B\sdiff B_\varepsilon) < \varepsilon\;\}.
\end{equation*}
\begin{lemma}
    If $A \in \mathcal{G}$, then $A^c \in \mathcal{G}$.
\end{lemma}
\begin{proof}
    Let $A \in \mathcal{G}$, and let $\varepsilon > 0$. Then choose $B_\varepsilon \in \mathcal{A}$
    such that $\mu(A \sdiff B_\varepsilon) < \varepsilon$. Hence $\mu(A^c \sdiff B^c_\varepsilon) < \varepsilon$,
    and since $B^c_\varepsilon \in \mathcal{A}$, we conclude $A^c \in \mathcal{G}$.
\end{proof}
\subsection*{(c)}
\begin{lemma}
    Let $A_n \in \mathcal{G}$,$n \geq 1$, with $A_1 \subseteq A_2 \subseteq \cdots \subseteq A_n \subseteq \cdots$. 
    If $A = \bigcup_{n\geq 1} A_n$, then 
    for any $\varepsilon > 0$ there exists $N > 0$ such that $\mu(A\sdiff A_N) < \varepsilon$.
\end{lemma}
\begin{proof}
    We compute,
    \begin{align*}
        \mu(A) &= \mu(\bigcup_{n=1}^\infty A_n)\\
        &= \lim_{n\rightarrow\infty} \mu(A_n).
    \end{align*}
    And, 
    \begin{equation*}
        \mu(A\sdiff A_n) = \mu(A\setminus A_n) = \mu(A)-\mu(A_n).
    \end{equation*}
    Hence,
    \begin{equation*}
        \lim_{n\rightarrow\infty} \mu(A\sdiff A_n) = 0.
    \end{equation*}
\end{proof}
\subsection*{(d)}
\begin{corollary}
    $A \in \mathcal{G}$
\end{corollary}
\begin{proof}
    Let $\varepsilon > 0$. Choose $N > 0$ such that $\mu(A \sdiff A_n) < \varepsilon/2$
    and select $B \in \mathcal{A}$ such that $\mu(B \sdiff A_n) < \varepsilon/2$.
    Hence,
    \begin{equation*}
        \mu(A\sdiff B) \leq \mu(A\sdiff A_n \cup B\sdiff A_n) \leq \mu(A\sdiff A_n)+\mu(B\sdiff A_n) < \varepsilon.
    \end{equation*}
    Hence $A \in \mathcal{G}$.
\end{proof}
\subsection*{(e)}
\begin{theorem}
    $\mathcal{G} = \mathcal{F}$.
\end{theorem}
\begin{proof}
    Suppose $A,B \in \mathcal{G}$. Then 
    choose $A_\varepsilon,B_\varepsilon \in \mathcal{A}$ such that
    $\mu(A\sdiff A_\varepsilon) < \varepsilon/2$ and $\mu(B\sdiff B_\varepsilon) < \varepsilon/2$.
    
    Then 
    \begin{align*}
        \mu((A\cap B)\sdiff (A_\varepsilon\cap B_\varepsilon)) &= \mu((A\cap B)\setminus(A_\varepsilon \cap B_\varepsilon)\cup((A_\varepsilon\cap B_\varepsilon)\setminus (A\cap B))\\
        &= \mu((A\cap B)\setminus A_\varepsilon \cup (A\cap B)\setminus B_\varepsilon \cup (A_\varepsilon \cap B_\varepsilon)\setminus A \cup (A_\varepsilon \cap B_\varepsilon)\setminus B)\\
        &\leq \mu(A\sdiff A_\varepsilon \cap B\sdiff B_\varepsilon)\\
        &< \varepsilon.        
    \end{align*}
    Hence $\mathcal{G}$ is closed under intersection. Thus $\mathcal{G}$
    is closed under relative complement since it is closed
    under complement.

    Hence $\mathcal{G}$ is a $d$-class containing $\mathcal{A}$ since we have
    shown that it is closed under complement and increasing countable union, hence $d(\mathcal{A}) \subseteq \mathcal{G}$.
    But since $\mathcal{A}$ is an algebra, it is a $\pi$-class. Hence $\mathcal{F} \subseteq \mathcal{G}$
    since $\sigma(\mathcal{A}) = d(\mathcal{A})$ by the monotone class theorem.  
    
    Since by definition $\mathcal{G} \subseteq \mathcal{F}$,
    we conclude $\mathcal{F} = \mathcal{G}$.  
\end{proof}

\section*{Question 3}
In this question we consider the measure space $(X,\mathcal{A},\mu)$
and the completed measure $\overline{\mu}$ with associated
algebra $\mathcal{A}_\mu$.
\begin{lemma}
    $\mathcal{A}_\mu$ is a $\sigma$-algebra on $X$.    
\end{lemma}
\begin{proof}
    Since $\mathcal{A} \subseteq \mathcal{A}_\mu$, we have $X \in \mathcal{A}_\mu$.
    
    Suppose $A \in \mathcal{A}_\mu$. Then by definition there are $E,F \in \mathcal{A}$
    with $E \subseteq A \subseteq F$ and $\mu(F \setminus E) = 0$. Hence we 
    have $F^c \subseteq A^c \subseteq E^c$, and $\mu(E^c \setminus A^c) = \mu(F\setminus E) = 0$.
    Since $F^c, E^c \in \mathcal{A}$, we conclude that $A^c \in A_\mu$.
    
    Now let $\{A_n\}_{n=1}^\infty$ be a countable subcollection of $\mathcal{A}_\mu$. 
    Choose $E_n, F_n \in \mathcal{A}$ for each $n\geq 1$ such that $E_n \subseteq A_n \subseteq F_n$
    and $\mu(F_n\setminus E_n) = 0$.
    Hence,
    \begin{equation*}
        \bigcup_{n=1}^\infty E_n \subseteq \bigcup_{n=1}^\infty A_n \subseteq \bigcup_{n=1}^\infty F_n
    \end{equation*}
    where the left and right hand sides are in $\mathcal{A}$, and
    \begin{equation*}
        \mu(\bigcup_{n=1}^\infty F_n \setminus \bigcup_{n=1}^\infty E_n) \leq \sum_{n=1}^\infty \mu(F_n\setminus E_n) = 0.
    \end{equation*}
    Hence $\mathcal{A}_\mu$ is a $\sigma$-algebra.    
\end{proof}

\begin{lemma}
    $\overline{\mu}$ is a measure on $(X,\mathcal{A}_\mu)$.
\end{lemma}
\begin{proof}
    We need to prove that $\mu$ is countably additive on $\mathcal{A}_\mu$. Suppose
    
    that $\{A_n\}_{n=1}^\infty$ is a sequence of $\overline{\mu}$-measurable sets
    that is pairwise disjoint. Then choose $E_n,F_n \in \mathcal{A}$
    such that $E_n \subseteq A_n \subseteq F_n$ and $\mu(F_n\setminus E_n) = 0$. 
    
    Hence $\overline{\mu}(A_n) = \mu(E_n)$. 
    Then we have
    \begin{equation*}
        \bigcup_{n=1}^\infty  E_n \subseteq\bigcup_{n=1}^\infty A_n \subseteq  \bigcup_{n=1}^\infty F_n
    \end{equation*}
    Then since ${\mu}(\bigcup_{n=1}^\infty F_n\setminus \bigcup_{n=1}^\infty E_n) = 0$,
    we have
    \begin{equation*}
        \overline{\mu}(\bigcup_{n=1}^\infty A_n) = \mu(\bigcup_{n=1}^\infty E_n).
    \end{equation*} 
    But since the $A_n$ are pairwise disjoint, so are the $E_n$, and so
    \begin{align*}
        \overline{\mu}(\bigcup_{n=1}^\infty A_n) &= \sum_{n=1}^\infty \mu(E_n)\\ 
        &= \sum_{n=1}^\infty \overline{\mu}(A_n).
    \end{align*}
    
    Hence $\overline{\mu}$ is countably additive, hence a measure.
   
\end{proof} 

\begin{lemma}
    The restriction of $\overline{\mu}$ to $\mathcal{A}$
    is $\mu$.
\end{lemma}
\begin{proof}
    Let $E \in \mathcal{A}$. Then $E \in \mathcal{A}_\mu$
    since $E\subseteq E\subseteq E$, and $\mu (E\setminus E) = 0$.
    
    Then $\overline{\mu} = \mu(E)$. Hence $\overline{\mu}$
    restricted to $\mathcal{A}$ is $\mu$.
\end{proof}

\begin{lemma}
    The measure space $(X,\mathcal{A}_\mu,\overline{\mu})$
    is complete.
\end{lemma}
\begin{proof}
    We need to show that every subset of a $\overline{\mu}$-null
    set is measurable.
    
    Suppose $E \in 2^X$ with $E \subseteq F \in \mathcal{A}_\mu$
    and $\overline{\mu}(F) = 0$. Then since $\emptyset \in \mathcal{A}$,
    and there is some set $B \in \mathcal{A}$ with $F \subseteq B$
    and $\mu(B\setminus \emptyset) = \mu(B) = 0$,
    then $F \in \mathcal{A}_\mu$ and $\overline{\mu}(F) = \mu(B) = 0$.
    
    Hence $(X,\mathcal{A}_\mu, \overline{\mu})$ is complete.
\end{proof} 
\end{document}
