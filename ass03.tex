\documentclass{unswmaths}

\usepackage{unswshortcuts}

\begin{document}

\subject{Measure Theory}
\author{Edward McDonald}
\title{Homework 3}
\studentno{3375335}


\setlength\parindent{0pt}


\newcommand{\Bor}{\mathcal{B}(\Rl)}
\newcommand{\sdiff}{\bigtriangleup}

\unswtitle{}

\section*{Question 1}
In this question we work over $\Rl^d$, with $d > 1$. $\lambda_d$ 
is the Lebesgue measure on $\Rl^d$.
\begin{theorem}
    Let $\ell \subset \Rl^d$ be a line. Then $\lambda_d(\ell) = 0$.
\end{theorem}   
\begin{proof}
    Since $\ell$ is closed, $\ell$ is Borel and hence Lebesgue measurable.
    
    Since $\lambda_d$ is translation invariant, we may assume that 
    $0 \in \ell$. Let $v \in \ell$ with $v \neq 0$. Consider half open line segment,
    \begin{equation*}
        S = \{tv\;:\;t \in [0,1)\;\}.
    \end{equation*}
    $S$ is Borel, hence measurable. Suppose $v = (v_1,v_2,\ldots,v_d)$, and define the box,
    \begin{equation*}
        B_n = \prod_{k=1}^d [0,\frac{v_k}{2^{n}}).
    \end{equation*}
    If $t \in [0,1)$ then for any $n > 0$ there exists
    an integer $k$ such that $\frac{k}{2^n} \leq v \leq \frac{k+1}{2^n}$.
    
    Hence,
    \begin{equation*}
        S \subset \bigcup_{k=0}^{2^n-1} (B_n+\frac{k}{2^{n}}v).
    \end{equation*}
    So by translation invariance,
    \begin{equation*}
        \lambda_d(S) \leq 2^n \lambda_d(B_n).
    \end{equation*}
    But by definition, $\lambda_d(B_n) = \frac{1}{2^{nd}}\lambda_d(B_0)$.
    Hence, 
    \begin{equation*}
        \lambda_d(S) \leq 2^{n(1-d)}\lambda_d(B_0).
    \end{equation*}
    Since $d > 1$, $n$ is arbitrary and $\lambda_d(B_0)$ is finite, we
    conclude that $\lambda_d(S) = 0$.
    
    Now since
    \begin{equation*}
        \ell = \bigcup_{n\in \mathbb{Z}} (nv+S)
    \end{equation*}
    by translation invariance we conclude that $\lambda_d(\ell) = 0$.    
\end{proof}

\section*{Question 2}
Let $A$,$B$ and $C$ be sets, and define $A \sdiff B := (A\setminus B)\cup (B\setminus A)$.
\subsection*{(a)}
\begin{lemma}
    
\end{lemma}

\subsection*{(b)}
Now we define
\begin{equation*}
    \mathcal{G} := \{\;B \in \mathcal{F}\;:\;\forall\varepsilon > 0\;\exists B_\varepsilon \in \mathcal{A}\;\text{such}\;\text{that}\;\mu(B\sdiff B_\varepsilon) < \varepsilon\;\}.
\end{equation*}
\begin{lemma}
    If $A \in \mathcal{G}$, then $A^c \in \mathcal{G}$.
\end{lemma}
\begin{proof}
    Let $A \in \mathcal{G}$, and let $\varepsilon > 0$. Then choose $B_\varepsilon \in \mathcal{A}$
    such that $\mu(A \sdiff B_\varepsilon) < \varepsilon$. Hence $\mu(A^c \sdiff B^c_\varepsilon) < \varepsilon$,
    and since $B^c_\varepsilon \in \mathcal{A}$, we conclude $A \in \mathcal{G}$.
\end{proof}
\subsection*{(c)}
\begin{lemma}
    Let $A_n \in \mathcal{G}$,$n \geq 1$, with $A_1 \subseteq A_2 \subseteq \cdots A_n \cdots$. 
    If $A = \bigcup_{n\geq 1} A_n$, then 
    for any $\varepsilon > 0$ there exists $N > 0$ such that $\mu(A\sdiff A_N) < \varepsilon$.
\end{lemma}
\begin{proof}
    We compute,
    \begin{align*}
        \mu(A) &= \mu(\bigcup_{n=1}^\infty A_n)\\
        &= \lim_{n\rightarrow\infty} \mu(A_n).
    \end{align*}
    And, 
    \begin{equation*}
        \mu(A\sdiff A_n) = \mu(A\setminus A_n) = \mu(A)-\mu(A_n).
    \end{equation*}
    Hence,
    \begin{equation*}
        \lim_{n\rightarrow\infty} \mu(A\sdiff A_n) = 0.
    \end{equation*}
\end{proof}
\subsection*{(d)}
\begin{corollary}
    $A \in \mathcal{G}$
\end{corollary}
\begin{proof}
    Let $\varepsilon > 0$. Choose $N > 0$ such that $\mu(A \sdiff A_n) < \varepsilon/2$
    and select $B \in \mathcal{A}$ such that $\mu(B \sdiff A_n) < \varepsilon/2$.
    Hence,
    \begin{equation*}
        \mu(A\sdiff B) \leq \mu(A\sdiff A_n \cup B\sdiff A_n) \leq \mu(A\sdiff A_n)+\mu(B\sdiff A_n) < \varepsilon.
    \end{equation*}
    Hence $A \in \mathcal{G}$.
\end{proof}
\subsection*{e}
\begin{theorem}
    $\mathcal{G} = \mathcal{F}$.
\end{theorem}
\begin{proof}
    $\mathcal{G}$ is a $d$-class containing $\mathcal{A}$, hence $d(\mathcal{A}) \subseteq \mathcal{G}$.
    But since $\mathcal{A}$ is an algebra, it is a $\pi$-class. Hence $\mathcal{F} \subseteq \mathcal{G}$
    since $\sigma(\mathcal{A}) = d(\mathcal{A})$ by the monotone class theorem.    
\end{proof}

\section*{Question 3}
In this question we consider the measure space $(X,\mathcal{A},\mu)$
and the completed measure $\overline{\mu}$ with associated
algebra $\mathcal{A}_\mu$.
\begin{lemma}
    $\mathcal{A}_\mu$ is a $\sigma$-algebra on $X$.    
\end{lemma}
\begin{proof}
    Since $\mathcal{A} \subseteq \mathcal{A}_\mu$, we have $X \in \mathcal{A}_\mu$.
    
    Suppose $A \in \mathcal{A}_\mu$. Then by definition there are $E,F \in \mathcal{A}$
    with $E \subseteq A \subseteq F$ and $\mu(F \setminus E) = 0$. Hence we 
    have $F^c \subseteq A^c \subseteq E^c$, and $\mu(E^c \setminus A^c) = \mu(F\setminus E) = 0$.
    Since $F^c, E^c \in \mathcal{A}$, we conclude that $A^c \in A_\mu$.
    
    Now let $\{A_n\}_{n=1}^\infty$ be a countable subcollection of $\mathcal{A}_\mu$. 
    Choose $E_n, F_n \in \mathcal{A}$ for each $n\geq 1$ such that $E_n \subseteq A_n \subseteq F_n$
    and $\mu(F_n\setminus E_n) = 0$.
    Hence,
    \begin{equation*}
        \bigcup_{n=1}^\infty E_n \subseteq \bigcup_{n=1}^\infty A_n \subseteq \bigcup_{n=1}^\infty F_n
    \end{equation*}
    where the left and right hand sides are in $\mathcal{A}$, and
    \begin{equation*}
        \mu(\bigcup_{n=1}^\infty F_n \setminus \bigcup_{n=1}^\infty E_n) \leq \sum_{n=1}^\infty \mu(F_n\setminus E_n) = 0.
    \end{equation*}
    Hence $\mathcal{A}_\mu$ is a $\sigma$-algebra.    
\end{proof}

\begin{lemma}
    $\overline{\mu}$ is a measure on $(X,\mathcal{A}_\mu)$.
\end{lemma}
\begin{proof}
    We need to prove that $\mu$ is countably additive on $\mathcal{A}_\mu$. Suppose
    
   
\end{proof} 

\end{document}
