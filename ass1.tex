\documentclass{unswmaths}

\usepackage{unswshortcuts}

\begin{document}

\subject{Measure Theory}
\author{Edward McDonald}
\title{Assignment 1}
\studentno{3375335}


\setlength\parindent{0pt}


\newcommand{\Bor}{\mathcal{B}(\Rl)}
\newcommand{\sdiff}{\bigtriangleup}

\unswtitle{}


\section*{Question 1}
Let $d \geq 1$ be an integer, and $S = \prod_{k=1}^d (a_k,b_k]$.
\begin{theorem}
    If $f:S\rightarrow \Rl$ is Riemann integrable, then $f$ is Lebesgue integrable
    and $\int_S f\;d\lambda = \int_S f(x)\;dx$.
\end{theorem}
\begin{proof}
    Choose a sequence of partitions $\mathcal{P}_n = \{C_{nk}\}$
    of $S$ such that the maximum diameter of $C_{nk}$ vanishes as $n$ goes to infinity,
    and $\mathcal{P}_{n+1}$ refines $\mathcal{P}_n$.
    Define the simple functions,
    \begin{align*}
        \ell_n &= \sum_{k} \inf_{x \in C_{nk}}f(x)\chi_{C_{nk}}\\
        u_n &= \sum_{k} \sup_{x \in C_{nk}}f(x)\chi_{C_{nk}}.
    \end{align*}
    By assumption these suprema and infima exist since $f$ is Riemann integrable.
    Then since the partition consists of disjoint sets, we may write
    \begin{align*}
        \int \ell_n\;d\lambda &= \sum_{k} \inf_{x \in C_{nk}} f(x)\lambda(C_{nk}) = L(f;\mathcal{P}_n)\\\
        \int u_n\;d\lambda &= \sum_k \sup_{x \in C_{nk}}f(x) \lambda(C_{nk}) = U(f;\mathcal{P}_n).
    \end{align*}
    By the assumption that $f$ is Riemann integrable, these integrals are finite.
    
    Since $\mathcal{P}_{n+1}$ refines $\mathcal{P}_n$, we must have that $\ell_{n+1}\geq \ell_n$
    and $u_{n+1}\leq u_n$. Hence the sequences $\{\ell_n\}_{n=1}^\infty$ and $\{u_n\}_{n=1}^\infty$
    are monotonically increasing and decreasing respectively.
    
    Since for each $x \in S$, we have $\ell_n(x) \leq u_k(x) < \infty$ for all $k$, the supremum $\sup_n \ell_n$
    is a well defined function, and is measurable, and we have the bound $\sup_n \ell_n \leq u_k$ for all $k$.
    Similarly, $\inf_n u_n$ is well defined and measurable, and we have $\sup_n \ell_n \leq \inf_n u_n$.
    
    For all $n$, the we have the bound $|\ell_n| \leq \max\{|\ell_1|,|u_1|\} \in L^1$
    and $|u_n| \leq \max\{|u_1|,|\ell_1|\} \in L^1$,    
    so by the dominated
    convergence theorem,
    \begin{equation*}
        \lim_{n\rightarrow\infty} \int_S \ell_n\;d\lambda = \int_S \sup_{n} \ell_n\;d\lambda \leq \int_S \inf_{n} u_n\;d\lambda = \lim_{n\rightarrow\infty}\int_Su_n\;d\lambda.
    \end{equation*}
    However the left and right hand sides must be equal since they are the limit of the lower and upper Riemann sums respectively. Hence,
    \begin{equation*}
        \int_S |\inf_n u_n-\sup_n \ell_n|\;d\lambda = 0.
    \end{equation*}
    Thus, $\inf_n u_n = \sup_n \ell_n$ $\lambda$-almost everywhere.
    
    Note that for any $x$, $\inf_n u_n(x) \geq f(x) \geq \sup_n \ell_n(x)$.
    
    Thus $f(x) = \inf_n u_n(x) = \sup_n \ell_n(x)$ for $\lambda$-almost all $x$.
    
    Hence, $f$ agrees almost everywhere with a Lebesgue integrable function, so we may
    regard it as Lebesgue integrable, and
    \begin{equation*}
        \int_S f\;d\lambda = \lim_{n\rightarrow\infty} \int_S \ell_n\;d\lambda = \int_S f(x)\;dx.
    \end{equation*}
    
\end{proof}

\begin{theorem}
    $f$ is continuous $\lambda$-almost everywhere on $S$.
\end{theorem}
\begin{proof}
    Choose a specific sequence of partitions of $S$, $\mathcal{P}_n = \{C_{nk}\}_k$
    such that each $C_{nk}$ is a box. Let $E$ denote the set of points
     $x \in S$ such that $\lim_n u_n(x) = f(x) = \lim_n \ell_n(x)$
     and also such that $x$ does not lie on the boundary of $C_{nk}$ for any $n$ or $k$.
     
     We know that $\lambda(S\setminus E) = 0$, and $E$ is Borel measurable.
     
    Let $\varepsilon > 0$, and $x \in E$. Choose $N$ large enough such that for $n > N$,
    $u_n(x) -\ell_n(x) < \varepsilon$. Since $x$ lies in the interior of some $C_{Nk}$, choose $\delta$
    small enough such that $\|y-x\| < \delta$ implies that $y \in C_{Nk}$.
    
    Since $u_N$ and $\ell_N$ are constant on $C_{Nk}$, we have $\ell_N(x) \leq f(y) \leq u_N(x)$. 
    Hence $|f(x)-f(y)| < \varepsilon$.
\end{proof}


\section*{Question 2}
Let $X$ be a set, and $(Y,\mathcal{B})$ is a measurable space. $f:X\rightarrow Y$
is a function.
\subsection*{(a)}
\begin{lemma}
    The collection of subsets of $X$, $\mathcal{A} = \{f^{-1}(B)\;:\;B \in \mathcal{B}\}$
    is a $\sigma$-algebra and the smallest $\sigma$-algebra on $X$
    such that $f$ is measurable.
\end{lemma}
\begin{proof}
    Clearly $X = f^{-1}(Y)$ and $\emptyset = f^{-1}(\emptyset)$ are in $\mathcal{A}$,
    and if $A = f^{-1}(B) \in \mathcal{A}$ then $A^c = f^{-1}(B^c) \in \mathcal{A}$.
    
    
    Now if $\{A_n\}_{n=1}^{\infty} \subseteq \mathcal{A}$, with $A_n = f^{-1}(B_n)$,
    then $\bigcup_n A_n = f^{-1}(\bigcup_n B_n) \in \mathcal{A}$.
    
    Hence $\mathcal{A}$ is a $\sigma$-algebra.
    
    If $\mathcal{A}'$ is any other $\sigma$-algebra on $X$ such that $f$
    is $\mathcal{A}'/\mathcal{B}$ measurable, then by definition
    for any $B \in \mathcal{B}$, $f^{-1}(B) \in \mathcal{A}'$, so    
    we
    must have $\mathcal{A} \subseteq \mathcal{A}'$.
        
    
\end{proof}
\subsection*{(ii)}
Now let $(Z,\mathcal{C})$
be a measurable space, and $h:(X,\mathcal{A})\rightarrow (Z,\mathcal{C})$
is measurable and takes only countably many values $\{c_n\}_{n=1}^\infty \subseteq \mathcal{C}$.

\begin{lemma}
    There exists a measurable function $g:(Y,\mathcal{B})\rightarrow (Z,\mathcal{C})$
    such that $h = g\circ f$.
\end{lemma}
\begin{proof}
    Since for each $n$, $\{c_n\} \in \mathcal{C}$, by the measurability
    of $h$ we have $h^{-1}(\{c_n\}) \in \mathcal{A}$.
    
    Hence there exist $B_n \in \mathcal{B}$ such that $h^{-1}(\{c_n\}) = f^{-1}(B_n)$.
    
    Define $g:Y\rightarrow Z$ by $g(x) = c_n$ if $x \in B_n$. 
    
    This is unambiguous, since if $B_n \cap B_m \neq \emptyset$, then $c_n = c_m$,
    
    and $h^{-1}(\{c_n\}_{n=1}^\infty) = X = f^{-1}(\bigcup_{n} B_n)$,
    so $g(x)$ is uniquely defined for any $x \in Y$.
    
    By construction $g$ is measurable, since if $C \in \mathcal{C}$
    contains the points $\{c_{n_k}\}_{k=1}^\infty$, then $g^{-1}(C) = \bigcup_{n_k} B_{n_k} \in \mathcal{B}$.
    
    If $x \in X$, then $f(x) \in B_n$ for some $n$ since $X = f^{-1}(\bigcup_n B_n)$.
    Then $g(f(x)) = c_n$, but since $f(x) \in B_n$, $x \in f^{-1}(B_n) = h^{-1}(\{c_n\})$.
    
    So $h(x) = c_n = g(f(x))$.
    
    Hence we have $h = g\circ f$, as required.
\end{proof}

    
\section*{Question 3}
    In this question, $(X,\mathcal{A},\mu)$ is a measure space.
    
    Suppose $\{A_{n}\}{n\geq 0}$ is a sequence of sets in $\mathcal{A}$ then 
    the following holds:
    \begin{lemma}
    \label{charSup}
        \begin{equation*}
            \inf_{n} \chi_{A_n} = \chi_{\bigcap_n A_n}
        \end{equation*}
        and
        \begin{equation*}
            \sup_{n} \chi_{A_n} = \chi_{\bigcup_n A_n}
        \end{equation*}
    \end{lemma}
    \begin{proof}
        Let $x \in X$. If $\inf_n \chi_{A_n}(x) = 1$ means that
        there is no $k$ such that $x \notin A_n$. Hence $x \in \bigcap_{n} A_n$.
        
        Similarly, if $x \in \bigcap_{n} A_n$, then $\inf_n \chi_{A_n}(x) = 1$
        since for any $n$, $\chi_{A_n}(x) = 1$.
        
        Now we write, using $\chi_{B^c} = 1-\chi_B$,
        \begin{align*}
            \inf_n \chi_{A_n^c} &= \chi_{\bigcup_n A_n^c}\\
            1-\inf_n\chi_{A_n^c} &= \chi_{\bigcap_n A_n}\\
            \sup_n 1-\chi_{A_n^c} &=  \chi_{\bigcap_n A_n}\\
            \sup_n \chi_{A_n} &= \chi_{\bigcap_n A_n}
        \end{align*}
    \end{proof}
    
    Now we define
    \begin{equation*}
        \liminf_n A_n := \bigcup_n \bigcap_{k\geq n} A_k.
    \end{equation*}
    
    \begin{theorem}
    \label{3equiv}
        The following are equivalent,
        \begin{align*}
            x &\in \liminf_n A_n \\
            \liminf_n \chi_{A_n}(x) &= 1
        \end{align*}
        and $x \in A_n$ for all but finitely many $n$.
    \end{theorem}
    \begin{proof}
        Using lemma \ref{charSup}, we write 
        \begin{align*}
            \liminf_n \chi_{A_n}(x) &= \sup_{n} \inf_{k\geq n} \chi_{A_k}\\
            &= \sup_{n} \chi_{\bigcap_{k\geq n}} A_k\\
            &= \chi_{\liminf_n A_n}(x).
        \end{align*}
        
        
    Hence $x \in \liminf_n A_n$ if and only if $\liminf_n \chi_{A_n} = 1$.
    
    If $\liminf_n \chi_{A_n}(x) = 1$. then $1$ is the only limit point
    of the sequence $\chi_{A_n}(x)$, hence since $\chi_{A_n}$
    takes only the values $0$ and $1$, it must take the value $0$
    only finitely many times. Hence $x \in A_n$ for all but finitely many $n$.
    
    Conversely, if $x \in A_n$ for all but finitely many $n$, 
    then the numerical sequence $\chi_{A_n}(x)$ takes the value $0$
    only finitely many times. Since it must have a limit point,
    we conclude $\liminf_{n}\chi_{A_n}(x) = 1$.
    \end{proof}
    
    Now we define
    \begin{equation*}
        \limsup_n A_n = \bigcap_n \bigcup_{k\geq n} A_k
    \end{equation*}
    
    \begin{theorem}
        The following are equivalent:
        \begin{align*}
            x &\in \limsup_n A_n\\
            \limsup_{n} \chi_{A_n}(x) &= 1
        \end{align*}
        and $x \notin A_n$ for infinitely many $n$.
    \end{theorem}
    \begin{proof}
        The equivalence of the first two statements is identical
        to theorem \ref{3equiv}.
        
        For the third statement, if $\limsup_n \chi_{A_n}(x) = 1$
        then the numerical sequence $\chi_{A_n}(x)$ has $1$ as a limit point,
        so $x$ must be in $A_n$ infinitely often.
        
        Conversely, if $x$ is in $A_n$ infinitely often then $1$
        is a limit point of the sequence $\chi_{A_n}(x)$. Hence it must be the largest
        limit point so $\limsup_{n} \chi_{A_n}(x) = 1$.
    \end{proof} 
    
    Hence it is clear that $\liminf_n A_n \subseteq \limsup_n A_n$
    since $\chi_{\liminf_n A_n}(x) = \liminf_{n} \chi_{A_n}(x) \leq \limsup_{n} \chi_{A_n}(x) = \chi_{\limsup_n A_n}$.
    
    
\section*{Question 4}
For this question, $(X,\mathcal{A},\mu)$ is a measure space with $\mu(X) <\infty$
and $C \subseteq X$.

\subsection*{(a)}
\begin{lemma}
    $\mathcal{A}_C = \{A\cap C\;:\;A \in \mathcal{A}\}$
    is a $\sigma$-algebra on $C$.
\end{lemma}
\begin{proof}
    We have $C = C\cap X \in\mathcal{A}_C$,
    and if $A\cap C \in \mathcal{A}$, then $A^c\cap C = C\setminus A \in \mathcal{A}_C$. 
    
    Now if $\{C_n\}_{n=1}^\infty$ is a countable subset of $\mathcal{A}_C$,
    with $C_n = C\cap A_n$ for $A_n \in \mathcal{C}$,
    then $\bigcup_n C_n = \bigcup_n C\cap A_n = C\cap \bigcup_n A_n \in \mathcal{A}_C$.
    
    Hence $\mathcal{A}_C$ is a $\sigma$-algebra.
\end{proof}

\subsection*{(b)}
Now define the outer and inner measures $\mu_*,\mu^*:2^X\rightarrow[0,\infty]$
by the usual formualae,
\begin{align*}
    \mu_*(B) &= \sup\{\mu(A)\;:\;A\subseteq B,A\in\mathcal{A}\}\\
    \mu^*(B) &= \inf\{\mu(A)\;:\;A\supseteq B,A\in\mathcal{A}\}.
\end{align*}
\begin{lemma}
    For each $B \in 2^X$ there exist $A_0 \subseteq B \subseteq A_1$
    with $A_0,A_1 \in \mathcal{A}$ and $\mu(A_0) = \mu_*(B)$
    and $\mu(A_1) = \mu^*(B)$.
\end{lemma}
\begin{proof}
    We may choose $C_n \in \mathcal{A}$ such that $\mu^*(B) \leq \mu(C_n) \leq \mu^*(B)+\frac{1}{n}$
    
    Then $B \subseteq \bigcap_n C_n \in \mathcal{A}$,
    and for every $n$,
    \begin{equation*}
        \mu(\bigcap_{n} C_n) \leq \mu^*(B) + \frac{1}{n}
    \end{equation*} 
    Hence, if $A_1 = \bigcap_n C_n$, we have $\mu(A_1) = \mu^*(B)$.
    
    
    Similarly, choose $C_n' \in \mathcal{A}$ $C_n' \subseteq B$
    and $\mu_*(B) \geq \mu(C_n') \geq \mu*(B) -\frac{1}{n}$,
    and define $A_0 = \bigcup_n C_n'$.
\end{proof}

\subsection*{(c)}
Let $C_1 \in \mathcal{A}$
with $C \subseteq C_1$
be such that $\mu(C_1) = \mu^*(C)$.

\begin{lemma}
    $\mu^*(C_1\setminus C) = 0$.
\end{lemma}
\begin{proof}
    There exists some $B \in \mathcal{A}$ with $B \subseteq C_1\setminus C$
    such that $\mu(B) = \mu^*(C_1\setminus C)$.
    If $\mu(B) > 0$, then $\mu(C_1) = \mu(B) + \mu(C_1\setminus B) > \mu(C_1\setminus B)$.
    
    But $C \subseteq C_1\setminus B$, so $\mu^*(C) \leq \mu(C_1\setminus B)$.
    Hence $\mu(C_1) > \mu^*(C)$, but this is a contradiction.    
\end{proof} 

\begin{lemma}
    If $A \in \mathcal{A}$, then $\mu(A\cap C_1) = \mu^*(A\cap C)$.
\end{lemma}
\begin{proof}
    Since $C \subseteq C_1$, we have $\mu^*(A\cap C) \leq \mu(A\cap C_1)$.
    
    Now by the subadditivity of $\mu^*$, $\mu(C_1\cap A) \leq \mu^*(C\cap A) + \mu^*((C_1\setminus C)\cap A)$.
    But $(C_1\setminus C)\cap A\subseteq C_1\setminus C$, so $\mu^*((C_1\setminus C)\cap A) = 0$.
    
    Hence $\mu(A\cap C_1) = \mu^*(A\cap C)$.
\end{proof}


\begin{theorem}
\label{indep}
    Suppose $A_1, A_2 \in \mathcal{A}$ with $A_1\cap C = A_2\cap C$. Then
    $\mu(A_1\cap C_1) = \mu(A_2\cap C_1)$.
\end{theorem}
\begin{proof}
    Since $A_1\cap C = A_2\cap C$, we 
    have $\mu(A_1\cap C_1) = \mu^*(A_1\cap C) = \mu^*(A_2\cap C) = \mu(A_2\cap C_1)$.

%    By the $\mu^*$ measurability of elements of $\mathcal{A}$, for any $B \subseteq X$, 
%    \begin{equation*}
%        \mu^*(B) = \mu^*(B\cap A_i\cap C_1) + \mu^*(B\setminus(A_i\cap C_1)).
%    \end{equation*}
%    For $i = 1,2$. Choose $B = C$, then we have
%    \begin{align*}
%        \mu(C_1) &= \mu^*(C)\\
%        &= \mu^*(A_i\cap C) + \mu^*(C\setminus(A_i\cap C_1))\\
%        &= \mu^*(A_i\cap C) + \mu^*(C\setminus A_i).
%    \end{align*}
%    
%    Now,
%    \begin{equation*}
%        \mu(C_1) = \mu(A_i\cap C_1)+\mu(C_1\setminus A_i).
%    \end{equation*}
%    
%    So we have,
%    \begin{equation*}
%        \mu(A_i\cap C_1) + \mu(C_1\setminus A_i) = \mu^*(A_i\cap C) + \mu^*(C\setminus A_i).
%    \end{equation*}
%    
%    However since $C\subseteq C_1$, we have $\mu^*(C\setminus A_i)\leq \mu^*(C_1\setminus A_i)$, 
%    
%    and by subadditivity of $\mu^*$, $\mu^*(C_1\setminus A_i) \leq \mu^*(C\setminus A_i) + \mu^*(C_1\setminus C)$. 
%    
%    Hence $\mu^*(C_1\setminus A_i) = \mu^*(C\setminus A_i)$.
%    
%    Thus $\mu(A_1\cap C_1) = \mu^*(A_1\cap C) = \mu*(A_2\cap C) = \mu(A_2\cap C_1)$.
\end{proof}


\subsection*{(c)}
Now define $\mu_C:\mathcal{A}_C\rightarrow[0,\infty]$
by $\mu_C(A\cap C) = \mu(A\cap C_1)$,

\begin{theorem}
    $\mu_C(B) = \mu^*(B)$ for every $B \in \mathcal{A}_C$.
\end{theorem}
\begin{proof}
    Let $B \in \mathcal{A}_C$, then $B = C\cap A$ for some $A \in \mathcal{A}$.
    
    Then $\mu_C(B) = \mu(A\cap C_1)$. This does not depend on the choice
    of $A$ by lemma \ref{indep}.
    
    Then $\mu_C(B) = \mu^*(A\cap C) = \mu^*(B)$.
    
%    Since $C \subseteq C_1$, we have $\mu^*(A\cap C) \leq \mu(A\cap C_1)$.
    
%    Since $A\cap C_1 = A\cap C \cup A\cap (C_1\setminus C)$,
%    hence $\mu(A\cap C_1)\leq \mu^*(A\cap C)+\mu^*(A\cap C_1\setminus C)$,
%    but $\mu^*(C_1\setminus C) = 0$, so $\mu(A\cap C_1)\leq \mu^*(A\cap C)$.
    
\end{proof}

\subsection*{(d)}

\begin{lemma}
\label{almostDisjointSequence}
    Suppose $(X,\mathcal{A},\mu)$ is now any measure space,
    and $\{A_n\}_{n=1}^\infty \subseteq \mathcal{A}$ is a sequence
    with $\mu(A_n\cap A_m) = 0$ for $n\neq m$. Then 
    \begin{equation*}
        \mu\left(\bigcup_{n=1}^\infty A_n\right) = \sum_{n=1}^\infty \mu(A_n).
    \end{equation*}
\end{lemma}
\begin{proof}
    We disjointify the sequence. Recursively define the sequence $\{B_n\}_{n=1}^\infty$
    by
    \begin{align*}
        B_1 &= A_1\\
        B_n &= A_n\setminus \left(\bigcup_{k=1}^{n-1} A_k\right), n>1.
    \end{align*}
    Then we have
    \begin{equation*}
        \bigcup_{n=1}^\infty B_n = \bigcup_{n=1}^\infty A_n.
    \end{equation*}
    And if $n > m$, $B_n\cap B_m = \emptyset$.
    
    Hence,
    \begin{equation*}
        \mu\left(\bigcup_{n=1}^\infty B_n\right) = \sum_{n=1}^\infty \mu(B_n).
    \end{equation*}
    
    Note that, 
    \begin{align*}
        \mu(A_n) &= \mu(B_n) + \mu(A_n\cap\bigcup_{k=1}^{n-1})\\
        &= \mu(B_n) + \mu(\bigcup_{k=1}^{n-1} A_n\cap A_k).
    \end{align*}
    
    But
    \begin{equation*}
        \mu(\bigcup_{k=1}^{n-1} A_n\cap A_k) \leq \sum_{k=1}^{n-1} \mu(A_n\cap A_k) = 0.
    \end{equation*}
    
    Hence $\mu(A_n) = \mu(B_n)$ so the result follows.  
    
\end{proof}

\begin{corollary}
    $\mu_C$ is a measure on $A_C$.
\end{corollary}
\begin{proof}
    We have proved that $\mu_C$ is the restriction of $\mu^*$
    to $\mathcal{A}_C$. It is therefore sufficient
    to prove that $\mu^*$ is countably additive on $\mathcal{A}_C$. 
    
    Let $\{B_n\}_{n=1}^\infty$ be a sequence of pairwise disjoint sets
    in $\mathcal{A}_C$. Let $B = \bigcup_n B_n$. Let $A_n \in \mathcal{A}$
    be such that $B_n = C\cap A_n$.
    
    Then,
    \begin{align*}
        \mu_C(B) &= \mu(C_1\cap\bigcup_{n=1}^\infty A_n)\\
        &= \mu(\bigcup_{n=1}^\infty C_1\cap A_n).
    \end{align*}
    The sets $\{C_1\cap A_n\}$ are not necessarily pairwise disjoint, however
    we may say that
    \begin{equation*}
        \mu(C_1\cap A_n \cap C_1\cap A_m) = 0
    \end{equation*}
    unless $n = m$.
    So by lemma \ref{almostDisjointSequence}, 
    \begin{equation*}
        \mu_C(B) = \sum_{n=1}^\infty \mu_C(B_n).
    \end{equation*}
    
    Hence $\mu_C$ is a measure.
    
%    Since we have $\mu_C(\emptyset) = \mu(\emptyset\cap C) = 0$,
%    it remains to prove that $\mu_C$ is countably additive.
%    
%    Let $\{B_n\}_{n=1}^\infty \subseteq \mathcal{A}_C$ be a pairwise
%    disjoint sequence. Let $B = \bigcup_n B_n$. Then $B_n = C\cap A_n$ for some $A_n \in \mathcal{A}$.
%    Then $\mu_C(B_n) = \mu(C_1\cap A_n)$. 
%    
%    Now $\mu_C(B) = \mu(\bigcup (C_1\cap A_n))$.
    
\end{proof}






\end{document}
