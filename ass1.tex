\documentclass{unswmaths}

\usepackage{unswshortcuts}

\begin{document}

\subject{Measure Theory}
\author{Edward McDonald}
\title{Assignment 1}
\studentno{3375335}


\setlength\parindent{0pt}


\newcommand{\Bor}{\mathcal{B}(\Rl)}
\newcommand{\sdiff}{\bigtriangleup}

\unswtitle{}


\section*{Question 2}
Let $X$ be a set, and $(Y,\mathcal{B})$ is a measurable space. $f:X\rightarrow Y$
is a function.
\begin{lemma}
    The collection of subsets of $X$, $\mathcal{A} = \{f^{-1}(B)\;:\;B \in \mathcal{B}\}$
    is a $\sigma$-algebra and the smallest $\sigma$-algebra on $X$
    such that $f$ is measurable.
\end{lemma}
\begin{proof}
    
\end{proof}

    
\section*{Question 3}
    In this question, $(X,\mathcal{A},\mu)$ is a measure space.
    
    Suppose $\{A_{n}\}{n\geq 0}$ is a sequence of sets in $\mathcal{A}$ then 
    the following holds:
    \begin{lemma}
    \label{charSup}
        \begin{equation*}
            \inf_{n} \chi_{A_n} = \chi_{\bigcap_n A_n}
        \end{equation*}
        and
        \begin{equation*}
            \sup_{n} \chi_{A_n} = \chi_{\bigcup_n A_n}
        \end{equation*}
    \end{lemma}
    \begin{proof}
        Let $x \in X$. If $\inf_n \chi_{A_n}(x) = 1$ means that
        there is no $k$ such that $x \notin A_n$. Hence $x \in \bigcap_{n} A_n$.
        
        Similarly, if $x \in \bigcap_{n} A_n$, then $\inf_n \chi_{A_n}(x) = 1$
        since for any $n$, $\chi_{A_n}(x) = 1$.
        
        Now we write, using $\chi_{B^c} = 1-\chi_B$,
        \begin{align*}
            \inf_n \chi_{A_n^c} &= \chi_{\bigcup_n A_n^c}\\
            1-\inf_n\chi_{A_n^c} &= \chi_{\bigcap_n A_n}\\
            \sup_n 1-\chi_{A_n^c} &=  \chi_{\bigcap_n A_n}\\
            \sup_n \chi_{A_n} &= \chi_{\bigcap_n A_n}
        \end{align*}
    \end{proof}
    
    Now we define
    \begin{equation*}
        \liminf_n A_n := \bigcup_n \bigcap_{k\geq n} A_k.
    \end{equation*}
    
    \begin{theorem}
    \label{3equiv}
        The following are equivalent,
        \begin{align*}
            x &\in \liminf_n A_n \\
            \liminf_n \chi_{A_n}(x) &= 1
        \end{align*}
        and $x \in A_n$ for all but finitely many $n$.
    \end{theorem}
    \begin{proof}
        Using lemma \ref{charSup}, we write 
        \begin{align*}
            \liminf_n \chi_{A_n}(x) &= \sup_{n} \inf_{k\geq n} \chi_{A_k}\\
            &= \sup_{n} \chi_{\bigcap_{k\geq n}} A_k\\
            &= \chi_{\liminf_n A_n}(x).
        \end{align*}
        
        
    Hence $x \in \liminf_n A_n$ if and only if $\liminf_n \chi_{A_n} = 1$.
    
    If $\liminf_n \chi_{A_n}(x) = 1$. then $1$ is the only limit point
    of the sequence $\chi_{A_n}(x)$, hence since $\chi_{A_n}$
    takes only the values $0$ and $1$, it must take the value $0$
    only finitely many times. Hence $x \in A_n$ for all but finitely many $n$.
    
    Conversely, if $x \in A_n$ for all but finitely many $n$, 
    then the numerical sequence $\chi_{A_n}(x)$ takes the value $0$
    only finitely many times. Since it must have a limit point,
    we conclude $\liminf_{n}\chi_{A_n}(x) = 1$.
    \end{proof}
    
    Now we define
    \begin{equation*}
        \limsup_n A_n = \bigcap_n \bigcup_{k\geq n} A_k
    \end{equation*}
    
    \begin{theorem}
        The following are equivalent:
        \begin{align*}
            x &\in \limsup_n A_n\\
            \limsup_{n} \chi_{A_n}(x) &= 1
        \end{align*}
        and $x \notin A_n$ for infinitely many $n$.
    \end{theorem}
    \begin{proof}
        The equivalence of the first two statements is identical
        to theorem \ref{3equiv}.
        
        For the third statement, if $\limsup_n \chi_{A_n}(x) = 1$
        then the numerical sequence $\chi_{A_n}(x)$ has $1$ as a limit point,
        so $x$ must be in $A_n$ infinitely often.
        
        Conversely, if $x$ is in $A_n$ infinitely often then $1$
        is a limit point of the sequence $\chi_{A_n}(x)$. Hence it must be the largest
        limit point so $\limsup_{n} \chi_{A_n}(x) = 1$.
    \end{proof} 
    
    Hence it is clear that $\liminf_n A_n \subseteq \limsup_n A_n$
    since $\chi_{\liminf_n A_n}(x) = \liminf_{n} \chi_{A_n}(x) \leq \limsup_{n} \chi_{A_n}(x) = \chi_{\limsup_n A_n}$.
    
    
    


\end{document}
